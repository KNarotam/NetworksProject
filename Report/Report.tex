\documentclass[10pt,twocolumn]{witseiepaper}

\usepackage{KJN}
\hyphenation{op-tical net-works semi-conduc-tor}
\usepackage{graphicx}
\usepackage{fancyhdr}
\usepackage{url}
\usepackage{amsmath}
\usepackage{listings}
\usepackage{algorithm}
\usepackage{algorithmic}
\usepackage{lipsum}
\usepackage{bookmark}

%----------------------------------------------------------------------------------------
%	PDF INFORMATION
%----------------------------------------------------------------------------------------
\ifpdf
\pdfinfo{
/Title (File Transfer Application)
/Author (Kishan Narotam (717 931) & Matthew van Rooyen (706 692))
/CreationDate (04/05/2019)
/Subject (ELEN4017)
/Keywords (ELEN4017, Networking, File Transfer Protocol)
}
\fi

%%%%%%%%%%%%%%%%%%%%%%%%%%%%%%%%%%%%%%%%%%%%%%%%%%%%%%%%%%%%%%%%%%%%%%%%%%%%%%%
\begin{document}

%----------------------------------------------------------------------------------------
%	TITLE
%----------------------------------------------------------------------------------------

\title{FILE TRANSFER APPLICATION\\ELEN4017: Network Fundamentals}

\author{Kishan Y Narotam (717 931) \& Matthew van Rooyen (706 692)
\thanks{School of Electrical \& Information Engineering, University of the
Witwatersrand, Private Bag 3, 2050, Johannesburg, South Africa}
}


%----------------------------------------------------------------------------------------
%	ABSTRACT
%----------------------------------------------------------------------------------------
\abstract{The purpose of this document is to provide an easy-to-use
template/style sheet to enable authors to prepare papers in the correct format
and style for the final year laboratory project. This document may be
downloaded from the School of Electrical and Information Engineering web site
and can be used as a template. To ensure conformity of appearance it is
essential that these instructions are followed. The abstract should be limited
to 50-200 words, which should concisely summarise the paper.}

\keywords{Four to six key words in alphabetical order, separated by commas.}


\maketitle
\thispagestyle{empty}\pagestyle{empty}


%----------------------------------------------------------------------------------------
%	MAIN BODY OF REPORT
%----------------------------------------------------------------------------------------
\section{INTRODUCTION}
\label{sec: Introduction}


%%%%%%%%%%%%%%%%%%%%%%%%%%%%%%%%%%%%%%%%%%%%%%%%%%%%%%%%%%%%%%%%%%%%%%%%%%%%%%%
%
\section{SYSTEM OVERVIEW}
\label{sec: System Overview}

\subsection{FTP Server}
\label{sec: FTP Server}
MATT

\subsection{FTP Client}
\label{sec: FTP Client}


%%%%%%%%%%%%%%%%%%%%%%%%%%%%%%%%%%%%%%%%%%%%%%%%%%%%%%%%%%%%%%%%%%%%%%%%%%%%%%%
%
\section{COMMANDS AND REPLIES}
\label{sec: Commands and Replies}

MATT, the table of our server's code implementations



%%%%%%%%%%%%%%%%%%%%%%%%%%%%%%%%%%%%%%%%%%%%%%%%%%%%%%%%%%%%%%%%%%%%%%%%%%%%%%%
%
\section{IMPLEMENTATION}
\label{sec: Implementation}


\subsection{Server Implementation}
\label{sec: Server Implementation}
MATT, see how Jared did this part of the report


\subsection{Client Implementation}
\label{sec: Client Implementation}

%%%%%%%%%%%%%%%%%%%%%%%%%%%%%%%%%%%%%%%%%%%%%%%%%%%%%%%%%%%%%%%%%%%%%%%%%%%%%%%
%
\section{DIVISION OF WORK}
\label{sec: Division of Work}

%%%%%%%%%%%%%%%%%%%%%%%%%%%%%%%%%%%%%%%%%%%%%%%%%%%%%%%%%%%%%%%%%%%%%%%%%%%%%%%
%
\section{RESULTS}
\label{sec: Results}


\subsection{Implemented server and client}
\label{sec: Results implemented server}

\subsection{FileZilla server and client}
\label{sec: Results FileZilla}


\subsection{Free online server and client}
\label{sec: Results Online}


%%%%%%%%%%%%%%%%%%%%%%%%%%%%%%%%%%%%%%%%%%%%%%%%%%%%%%%%%%%%%%%%%%%%%%%%%%%%%%%
%
\section{CRITICAL ANALYSIS}
\label{sec: Critical Analysis}

\subsection{Successes}
\label{sec: Successes}
MATT talk about server success

\subsection{Limitations and Problems}
\label{sec: Limitations and Problems}
MATT talkl about server issues and limitations

\subsection{Future Development}
\label{sec: Future Development}
MATT, say what could be improved


%%%%%%%%%%%%%%%%%%%%%%%%%%%%%%%%%%%%%%%%%%%%%%%%%%%%%%%%%%%%%%%%%%%%%%%%%%%%%%%
%
\section{CODE STRUCTURE AND PREREQUISITES}
\label{sec: CODE STRUCTURE AND PREREQUISITES}


\subsection{Code Structure}
\label{sec: Code Structure}

\subsection{System Prerequisites}
\label{sec: System Prerequisites}
MATT, just type the code needed to run your server

%%%%%%%%%%%%%%%%%%%%%%%%%%%%%%%%%%%%%%%%%%%%%%%%%%%%%%%%%%%%%%%%%%%%%%%%%%%%%%%
%
\section{CONCLUSION}
\label{sec: Conclusion}

A conclusion may review the main points of the paper, but do not replicate the
abstract as the conclusion.


%%%%%%%%%%%%%%%%%%%%%%%%%%%%%%%%%%%%%%%%%%%%%%%%%%%%%%%%%%%%%%%%%%%%%%%%%%%%%%%
%
%%%%%%%%%%%%%%%%%%%%%%%%%%%%%%%%%%%%%%%%%%%%%%%%%%%%%%%%%%%%%%%%%%%%%%%%%%%%%%%
%
\begin{thebibliography}{}

%**************************Section**************************%

\bibitem{Kopka}
H.~Kopka and P.~W. Daly, \emph{A Guide to \LaTeX}, 3rd~ed.\hskip 1em plus
  0.5em minus 0.4em\relax Harlow, England: Addison-Wesley, 1999.

\end{thebibliography}

%{\tiny \vfill \hfill \today \hspace{5mm} witseie-paper-2003.\TeX}

\end{document}

